\section*{Introduction}
\addcontentsline{toc}{section}{Introduction}

Le PJI (Projet individuel) se déroule dans le cadre de la première année de master informatique à l'Université de Lille 1. Le but est ici de développer un projet sur l'ensemble d'un semestre.

Ce rapport concerne le projet numéro 54, intitulé "Mais que font nos députés ? Une sociologie informatique du travail parlementaire" et encadré par messieurs Samuel Hym, enseignant chercheur au laboratoire CRIStAL de Lille, et Etienne Ollion, chercheur CNRS au laboratoire SAGE de Strasbourg.

Le but de ce projet est d'exploiter les comptes rendus intégraux de l'Assemblée nationale française afin de constituer des bases de données qui contiennent les informations des scrutins qui y sont votés : numéro, date, sujet du scrutin ainsi que nom, prénom, parti et vote des députés participants au scrutin.

Ce rapport présentera l'ensemble des travaux effectués sur le projet ainsi que leurs résultats. Pour commencer, le choix des outils utilisés sera exposé et justifié. Nous aborderons ensuite la manière dont les comptes rendus de l'Assemblée nationale ont été récupérés. Puis, nous verrons comment ces comptes rendus ont été filtrés et mis en forme afin de pouvoir en exploiter les données. Enfin, nous verrons comment les données ont été extraites et nettoyées afin de créer les bases.
