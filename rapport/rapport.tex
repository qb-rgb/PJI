\documentclass[a4paper,12pt,final]{article}
% Pour une impression recto verso, utilisez plutôt ce documentclass :
%\documentclass[a4paper,11pt,twoside,final]{article}

\usepackage[english,francais]{babel}
\usepackage[utf8]{inputenc}
\usepackage[T1]{fontenc}
\usepackage[pdftex]{graphicx}
\usepackage{setspace}

\usepackage[pdftex=true]{hyperref}

\usepackage[french]{varioref}
\usepackage{color}
\usepackage{amsmath}
\usepackage{amsthm}
\usepackage{amssymb}

\theoremstyle{definition}
\newtheorem*{exmp}{Exemple}

\newcommand{\reporttitle}{PJI (54) - Mais que font nos députés ? Une sociologie informatique du travail parlementaire} % Titre
\newcommand{\reportauthor}{Quentin \textsc{Baert}} % Auteur
\newcommand{\reportsubject}{Projet individuel} % Sujet
\newcommand{\HRule}{\rule{\linewidth}{0.5mm}}
\setlength{\parskip}{1ex} % Espace entre les paragraphes	

\makeatletter
\newcommand\verbfile[1]{%
	\begingroup
		\let\do\@makeother\dospecials
		\obeyspaces\obeylines\ttfamily
		\input#1\relax
	\endgroup
}
\makeatother

\hypersetup{
    pdftitle={\reporttitle},%
    pdfauthor={\reportauthor},%
    pdfsubject={\reportsubject},%
    pdfkeywords={rapport} {lille1} {master1} {Scala} {sociologie} {débutés}
}

\begin{document}
	% Inspiré de http://en.wikibooks.org/wiki/LaTeX/Title_Creation

\begin{titlepage}

\begin{center}

\includegraphics[scale=0.85]{images/38040_logo-trans.png}\\[2.5cm]

%\vspace{10mm}

\textsc{\Large \reportsubject}\\[0.5cm]
\HRule \\[0.4cm]
{\huge \reporttitle}\\[0.4cm]
\HRule \\[1.5cm]

%\begin{minipage}[t]{0.51\textwidth}
  %\begin{flushleft}
    %\includegraphics [width=50mm]{images/355_logo-lifl.png} \\[0.5cm]
   %\begin{spacing}{2}
      %\textsc{Université de Lille 1}
   %\end{spacing}
  %\end{flushleft}
%\end{minipage}
%\begin{minipage}[t]{0.48\textwidth}
  %\begin{flushright}
    %\includegraphics [width=50mm]{images/logo.jpg} \\[0.9cm]
    %\textsc{BioComputing}
  %\end{flushright}
%\end{minipage} \\[1.5cm]

\medskip

\begin{flushleft}
	\begin{tabular}{ll}
	\emph{Auteur :} & \reportauthor \\
	\emph{Encadrant universitaire :} & Samuel \textsc{Hym} \\
	\emph{Encadrant :} & Etienne \textsc{Ollion} \\
	\end{tabular}
	%\emph{Auteur :}
    %\reportauthor \\
    %\emph{Tuteur de stage :}
    %Guillaume \textsc{Madelaine}\\
    %\emph{Tuteur universitaire :}
    %Guillaume \textsc{Dubuisson Deplessis}
\end{flushleft}

%\begin{minipage}[t]{0.3\textwidth}
    %\emph{Auteur :}
    %\reportauthor \\
    %\emph{Tuteur de stage :}
    %Guillaume \textsc{Madelaine}\\
    %\emph{Tuteur universitaire :}
    %Guillaume \textsc{Dubuisson Deplessis}
%\end{minipage}
%\begin{minipage}[t]{0.6\textwidth}
  %\begin{flushright} \large
    %\emph{Tuteur de stage :} \\
    %\bfseries{Guillaume \textsc{Madelaine}} \\
    %\emph{Tuteur universitaire :} \\
    %\bfseries{Guillaume \textsc{Dubuisson Duplessis}}
  %\end{flushright}
%\end{minipage}

\vfill

{\large janvier à juin 2015}

\end{center}

\end{titlepage}

  	\cleardoublepage % Dans le cas du recto verso, ajoute une page blanche si besoin
  	\renewcommand{\contentsname}{Sommaire} % Dans le corps du document,avant la commande \tableofcontents.
	\tableofcontents % Table des matières
  	\sloppy          % Justification moins stricte : des mots ne dépasseront pas des paragraphes
  	\cleardoublepage
  	\section*{Remerciements}
\addcontentsline{toc}{section}{Remerciements}

Merci à Samuel Hym et Etienne Ollion de m'avoir encadré lors de la réalisation de ce projet, merci également pour leur accessibilité, leur bienveillance et pour avoir répondu à chacune de mes questions.

  	\cleardoublepage
  	\section*{Introduction}
\addcontentsline{toc}{section}{Introduction}

  	\cleardoublepage
  	\section{Choix des outils}

\subsection{Le langage Scala}

\cleardoublepage

\section{Récupération des comptes rendus intégraux de l'Assemblée nationale}

\subsection{Organisation des PDF sur le site de l'Assemblée nationale}

\subsection{Récupération des PDF}

\cleardoublepage

\section{Filtrage des données}

\subsection{Convertion des PDF en fichiers textes}

\subsection{Analyse des fichiers textes pour isoler ceux contenant des scrutins}

\cleardoublepage

\section{Mise en forme des informations récupérées}

\subsection{Représentation objet d'un scrutin}

\subsection{Extraction des données des scrutins}

\cleardoublepage

\section{Construction des bases de données}

\subsection{Bases de données simples}

\subsection{Bases de données détaillées}

\cleardoublepage


  	\cleardoublepage
  	\section*{Conclusion}
\addcontentsline{toc}{section}{Conclusion}

  	\cleardoublepage
  	\section*{Bibliographie/Webographie}

\subsection*{Bibliographie}

\begin{itemize}
\item[•] \textbf{Programming in Scala}, second edition.\newline Martin Odersky, Lex Spoon et Bill Venners, édition Broché, 2011
\vspace{0.4cm}
\item[•] \textbf{Scala Cookbook}.\newline Alvin Alexander, édition O'Reilly, 2013
\end{itemize}

\subsection*{Webographie}

\begin{itemize}
\item[•] \textbf{Lesson: Regular Expressions}\newline \url{http://docs.oracle.com/javase/tutorial/essential/regex/}
\end{itemize}

  	\cleardoublepage
  	\appendix

\end{document}

