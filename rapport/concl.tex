\section*{Conclusion}
\addcontentsline{toc}{section}{Conclusion}

Alors que les informations sur les scrutins tenus à l'Assemblée nationale sont disponibles dans un format facilement accessible pour les toutes dernières légsilatures de la cinquième République, cela est moins vrai pour les anciennes législatures.

Le but de ce projet était de récupérer l'ensemble des comptes rendus des dix premières législatures et de les utiliser afin de construire une base de données qui contienne toutes les données relatives aux scrutins qui s'y sont tenus. L'objectif est partiellement atteint si l'on considère que les scrutins ont correctement été isolés et que les données d'une petite partie des scrutins, les plus récents, sont d'ores et déjà rangées dans une base.

Il reste maintenant à étudier les documents mal traités pour affiner la récupération et le nettoyage des données. On pourra aussi envisager de changer de technique d'extraction du texte des PDFs ou même de pré-traiter le texte afin de le corriger avant d'essayer de l'analyser et de l'organiser.
