\section{Objectifs du projet}

L'objectif de ce projet est de collecter l'ensemble des scrutins publics tenus à l'Assemblée nationale entre 1958 et 2002 et de les organiser dans une base de données.
Les données initiales doivent être extraites de documents PDF, nettoyées et présentées sous forme d'une base facilement interrogeable (sous forme d'un fichier CSV par exemple).

Les différentes étapes sont donc de :
\begin{enumerate}
\item récupérer tous les PDF des comptes rendus intégraux sur le site de l'Assemblée nationale
\item convertir ces PDF en fichiers textes afin de pouvoir facilement travailler sur le texte des comptes rendus
\item filtrer les fichiers pour isoler ceux qui concernent des scrutins
\item créer des fichiers CSV à partir des fichiers qui contiennent en effet un ou plusieurs scrutins
\end{enumerate}

La mise en place et les résultats de ces différentes étapes sont présentés dans la suite de ce rapport.
