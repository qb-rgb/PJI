\section{Objectifs du projet}

L'objectif de ce projet était de collecter l'ensemble des scrutins publics tenus à l'Assemblée nationale entre 1958 et 2002 et de les organiser dans une base de données.
Les données initiales devaient être extraites de documents PDF, nettoyées et présentées sous forme d'une base de données facilement interrogeable (sous forme d'un fichier CSV par exemple).

Les différentes étapes étaient donc :
\begin{enumerate}
\item récupérer tous les PDF des comptes rendus intégraux sur le site de l'Assemblée nationale
\item convertir ces PDF en fichiers textes afin de pouvoir facilement travailler sur le texte des comptes rendus
\item tous les fichiers ne contenaient pas de scrutins, il fallait donc filtrer les fichiers pour isoler ceux qui nous intéréssaient
\item créer des fichiers CSV à partir des fichiers qui contenaient en effet un ou plusieurs scrutins
\end{enumerate}

Toutes ces étapes ont été traitées. Leur mise en place et leurs résultats seront présentés dans la suite de ce rapport.
